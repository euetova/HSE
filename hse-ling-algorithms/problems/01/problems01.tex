\documentclass[12pt,a4paper]{report}
% \textheight = 30cm
% \voffset = -36pt
\footskip = 0cm

\usepackage{cmap}
\usepackage{type1ec}
\usepackage[T2A]{fontenc}
\usepackage[utf8]{inputenc}
\usepackage[russian]{babel}

\usepackage{amsmath,amstext,amssymb}
\usepackage{fullpage}

\usepackage{ifpdf}
\ifpdf
  \usepackage[pdftex]{graphicx}
  \usepackage[pdftex,unicode,bookmarks=false]{hyperref}

  \pdfminorversion=5
  \pdfcompresslevel=9
  \pdfobjcompresslevel=9
\fi

\pagestyle{empty}


\renewcommand{\thesection}{\arabic{section}}
\renewcommand{\thesubsection}{}

\begin{document}

\begin{center}
\textbf{\LARGE{Теория алгоритмов. Листок 1}}\\
BSc, компьютерная лингвистика, НИУ ВШЭ\\
Выдан: 20 января 2016\\
Дедлайн: 27 января 00:00\\
\end{center}


\section{Асимптотические оценки функций}

\begin{enumerate}

  \item\,[4] Пользуясь определениями $\Omega$, $O$ и $\Theta$, докажите или опровергните следующие утверждения:
    \begin{enumerate}
      \item\,[1] $5n\log{n} + 3n + 1 \in O(n\log{n})$
      \item\,[1] $n^2 \in \Omega(n \log n)$
      \item\,[1] $f(n) \in O(g(n)) ~\Rightarrow~ g(n) \in O(f(n))$
      \item\,[1] $f(n) \in \Theta(g(n)) ~\Rightarrow~ g(n) \in \Theta(f(n))$
    \end{enumerate}

  \item\,[3] Покажите, что при любых вещественных $b > 0$ и $a$ выполняется
    $$
    (n+a)^b \in \Theta(n^b)
    $$

  \item\,[3] Покажите, что при любом $a > 0$
    $$
    \log n \in O(n^a)
    $$

  \item\,[7] На семинаре было введено асимптотическое сравнение функций:
    \begin{equation}
      \label{limit}
      \begin{gathered}
        \lim_{n\to\infty} \frac{f(n)}{g(n)} = 
        \begin{cases}
        0         &  \Rightarrow ~  f(n) \prec g(n) \\
        \infty    &  \Rightarrow ~  f(n) \succ g(n) \\
        < \infty  &  \Rightarrow ~  f(n) \asymp g(n) \\
        \end{cases}
      \end{gathered}
    \end{equation}
    И было дано сравнение часто встречающихся в анализе алгоритмов функций:
    $$1 ~~\prec~~ \log(n) ~~\prec~~ n ~~\prec~~ n\log(n) ~~\prec~~ n^2 ~~\prec~~ n^3 ~~\prec~~ c^n ~~\prec~~ n!$$%
    Пользуясь \eqref{limit} и методами вычисления пределов (правило Лопиталя, теорема о зажатой последовательности), докажите каждое из этих отношений. Вам так же придётся воспользоваться формулой Стирлинга:
    $$n! \sim \sqrt{2\pi n} \left(\frac{n}{e}\right)^n$$%

\end{enumerate}


\section{Оценка времени работы алгоритма}


\begin{enumerate}

  \item\,[3] Следующий код вычисляет многочлен $p(x) = \sum_{i=0}^{n} a_i x^i$:

  \begin{verbatim}
  def polynomial(a, n, x):
    p = a[0]
    xp = 1
    for i in range(1, n):
      xp = x * xp
      p += a[i] * xp
    return p\end{verbatim}%
  Оцените время работы алгоритма в худшем случае и в среднем. Найдите алгоритм, работающий быстрее, и реализуйте его на Python.


  \item\,[3] Докажите, что число итераций бинарного поиска по отсортированному массиву размера $n$ в худшем случае равно $\lfloor \log_2 n + 1 \rfloor$
  \item\,[7] Дано некоторое двоичное число $B$ фиксированной разрядности $n$. Необходимо определить номер разряда, в котором находится самый старший [ненулевой] бит числа. Например для $n=8$ и $B=01001101$ ответ будет равен 7.

  Самый простой способ -- проверять все биты начиная со старшего, пока не обнаружится <<1>>. В Python это выглядело бы так:
  \begin{verbatim}
  def msb(B, n):
      t = 1 << n
      for i in range(n, 0, -1):
          if B & t:
              return i
          t = t >> 1
      return 0\end{verbatim}%
  Пусть $n$ фиксировано, а $B$ -- равновероятно в диапазоне $[0, 2^n-1]$.
  Найдите {\em матожидание} числа итераций в цикле.

  {\em Подсказка:} $$\sum_{i=1}^{\infty} \frac{1}{2^i} = 1$$

\end{enumerate}



\section*{Литература}

\begin{enumerate}
  \item T.~Cormen et al. {\em Introduction to Algorithms}, Chapter 3
  \item Steven Skiena {\em The Algorithm Design Manual}, Chapter 2 + exercises
\end{enumerate}

\end{document}